\chapter{Introdução}
Neste capítulo, será explicado o escopo do trabalho - o que vai ser projetado -, quais resultados esperamos obter e por que o fazemos.
	
\section{Objetivo}
Este trabalho visa o projeto e implementação de um sistema capaz de obter a localização de macacos em reservas, dentre outros dados do ambiente ou do animal.

Está dentro do escopo deste projeto: a escolha das informações a serem colhidas dos animais; o método pelo qual elas serão obtidas, calculadas ou inferidas; a maneira como ela será transmitida, armazenada e apresentada; a escolha e implementação das tecnologias utilizadas; e a avaliação de desempenho sobre sua operação.

A composição do sistema prevê dispositivos embarcados inseridos em mochilinhas anexadas ao macaco, que, em conjunto, compõem uma rede de sensores para adquirir as informações necessárias do ambiente e dos indivíduos. Além disso, dispositivos de comunicação inseridos no ambiente ou manipulados por pesquisadores realizarão a coleta dessas informações, a fim de enviá-las para retenção em um servidor.  Este realiza o processamento e armazenamento dos dados que serão injetados em uma interface em software disponível para o usuário.

A discriminação do sistema é melhor realizada nos capítulos 4 e 6.

\section{Motivação}
Este trabalho visa atender às necessidade de monitoramento de saguis nas reservas do Instituto Butantã da Universidade de São Paulo. Dentre elas, destaca-se: a coleta de informações sobre os animais de forma simples e sem viés, tais como suas disposições em bando e suas temperaturas corporais; e a apresentação destas, de forma a facilitar análise de dados em  pesquisas acadêmicas e a manutenção da saúde dos animais.

Além disso, propõe-se que o este trabalho possa ser aplicado em qualquer reserva que necessite monitorar o comportamento de um bando de animais, sendo para tanto, veemente generalizado neste documento.

Ao final, espera-se obter um sistema que cubra o funcionamento mínimo de requisitos funcionais especificados no capítulo 4.

\section{Justificativa}
Animais silvestres são difíceis de serem observados em habitat natural por uma série de motivos. Primeiro, a partir do momento que o pesquisador se coloca no campo de visão do animal para observá-lo gera viés no comportamento deste, pois o animal também detecta a presença daquele e em muitas instâncias, age de forma irregular. Um caso específico se encontra no contexto de desamamentação de filhotes de macacos, cuja prática é realizada exclusivamente em um ambiente recluso e onde a presença de outrem é inadmissível, portanto o conhecimento sobre este comportamento é limitado para os pesquisadores.

Segundo, o auxílio tecnológico para essa tarefa é complicado uma vez que as tecnologias mais comumente usadas para monitoramento (câmeras de vídeo)  e rastreamento (GPS) são descartadas pela densidade da mata, que dificulta a observação, e imprecisão da informação obtida, que impede a inferência de comportamentos, respectivamente.

Assim, fica claro que, tanto para pesquisa laboratorial quanto para controle da localização com fim de manutenção da saúde de animais silvestres, o estudo para emancipação tecnológica para controle e automação se faz necessário.

Primordialmente, o conceito que move este projeto está atrelado ao que foi tratado por Handcock (2009) de que a interação social biológica revela preferências sociais e comportamentais. Por exemplo, é citado como o mapeamento de encontros entre machos e fêmeas pode correlacionar com acasalamento, o que possibilita estudos de emancipação genética em uma população.

Corolariamente, permeia ao projeto o conceito de Saúde Única (One Health) abordado por Zinsstag (2011), onde indissocia-se a visão de saúde humana, animal e do ambiente. Como do ponto de vista biológico o estudo do comportamento animal permite novas conclusões sobre seu comportamento em reservas e cativeiro, e sob o ponto de vista veterinário a análise dos dados sobre o animal e seu ambiente leva a melhoras na manutenção da saúde deste e do ambiente, sob ambos vê-se um impacto na saúde do homem. O sistema pode, por exemplo, ajudar na prevenção de febre amarela, por meio de coleta de dados em populações de animais alvos da doença.

\section{Organização do Trabalho}
O capítulo 2 deste trabalho relata os estudos realizados sobre conceitos fenotípicos e comportamentais do grupo de foco, sobre o contexto de pesquisa destes animais e sobre tecnologias voltadas para redes de sensores biológicos potencialmente válidas para este projeto.

O capítulo 3 trata da forma como o planejamento do sistema deverá ser pensado, enfatizando aspectos de projeto de sistemas embarcados. 

No capítulo 4 são indicados os requisitos funcionais e não funcionais levantados e algumas das possíveis soluções para essas problemáticas.

O capítulo 5 destrincha as tecnologias de fato selecionadas para serem utilizadas no sistema a ser implementado, considerando seus pontos falhos mas acentuando o motivo de terem sido escolhidas.

O capítulo 6, por sua vez, trabalha de fato a projeção do sistema seguindo todos os princípios estudados nos capítulos anteriores para que esteja clara a maneira de implementá-lo. Neste mesmo capítulo, os aspectos que tocam a implementação do sistema, que consideram a parte prática do objeto de estudo, também são descritos.

Por fim, no capítulo 7 são mostrados os resultados obtidos a partir do sistema desenvolvido através de validações e testes quantitativos e qualitativos de desempenho e satisfação.


